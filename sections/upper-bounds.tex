\section{Upper Bounds for the Extension Complexity}
% todo @ March 12

\begin{itemize}
  \item overview of upcoming section
\end{itemize}



\subsection{Arbitrary Convex n-gons}
% todo @ March 13 to 15

\begin{itemize}
  \item What is the key idea? 
  \begin{itemize}
    \item splitting n-gons into thin slices
    \item Extracting subset (vertices not neccessarily in order) with small extension, since $\xc(\conv(P \cup Q)) \leq \xc(P) + \xc(Q)$, repeatedly
    \item Constructing and glueing 3-d extended formulations: Theorem 28
  \end{itemize}
  \item essential statements
  \begin{itemize}
    \item Theorem 28
    \begin{itemize}
      \item What does it say?
      \item Maximum possible result with Theorem 28
      \item Why can't one just apply Theorem 28 (and skip the rest of the paper)?
    \end{itemize}
  \end{itemize}
  \item connections between statements
  \begin{itemize}
    \item (empty)
  \end{itemize}
  \item proofs worth repeating
  \begin{itemize}
    \item (empty)
  \end{itemize}
  \item Intuition for the technical part (especially definitions) through examples
\end{itemize}

\begin{tabular}{| p{50mm} | p{100mm} |}
  \hline
  \multicolumn{2}{|c|}{\textbf{Main results per chapter}}\\ \hline
  3. Preliminaries & Extension Complexity for union of polytopes \\ \hline
  4. Acute polyhedra & There is an acute polyhedron for a polygon, where all main edges except one have fixed direction \\ \hline
  5. Acute diagrams & There is an acute diagram for a polygon, where all main edges except one have fixed direction \\ \hline
  6. Glueing acute extensions together & Skip vertices of a polygon by joining acute diagrams to get an extended formulation which represents the original polygon \\ \hline
  7. Thin sequences & Splitting polygons is now a thing \& something about rays passing a horizontal line \\ \hline
  8. Good sequences & good sequences let you apply Theorem 28, identify good sequence via ray passing through top \\ \hline
  9. Extracting a slanted subsequence & When it's not slanted, it'll extend well (Lemma 49) \\ \hline
  10. Decreasing slanted sequences & Get estimates of xc for slanted sequences \\ \hline
  11. Extensions for slanted sequences & Subsequence of slanted sequence with good xc \\ \hline
  12. Completing the proof & Split the polygon, iteratively extract a subsequence and bound xc (see chapter 11 for slanted, chapter 9 for not slanted) \\ \hline
\end{tabular}



\subsection{Cyclic n-gons}
% todo @ March 16 to 18

\begin{itemize}
  \item overview of paper
  \item What is the key idea?
  \item coloring
  \item intuition of rescaling rows
  \item Higher-dimensional statements
\end{itemize}



\subsection{Comparison}
% todo @ March 19 to 20

\begin{itemize}
  \item general similarities (splitting of polygons, bulging of corners)
  \item general differences (splitting by angle vs count, coloring and independece of slices in general case)
  \item How is the circle property used? Can the approach be generalised?
  \item could the general case be $O(\sqrt{n})$, too? (current conjectures)
  \item Is there a way to adopt a proof to the other case?
\end{itemize}
