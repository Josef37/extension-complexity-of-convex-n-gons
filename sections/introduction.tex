\section{Introduction}

\subsection{Extension Complexity}

Six linear inequalities are required to describe a regular hexagon. However, the same polygon\footnote{We define a polygon to be a two-dimensional polytope, i.e.\ it is convex.} can be described by a polytope\footnote{A polytope is the convex hull of a finite set of points.} in $\R^3$ with five linear inequalities and a linear projection, see Figure~\ref{fig:hexagon}.

\begin{figure}[ht]
  \centering
  \includegraphics[scale=0.5,trim={0 2.5cm 0 2cm},clip]{hexagon}
  \caption{A regular hexagon as a projection of a three-dimensional polytope with five facets. \cite[Figure~1]{kwan2020extension}}
  \label{fig:hexagon}
\end{figure}

This is the key idea behind extended formulations. Simplifying the representation of polygons by projecting simpler, higher-dimensional polytopes ``down'' to the desired polytope.
One can think of it as ``compressing'' the representation. This mental image is inspired by the fact, that regular $n$-gons (i.e.\ polygons with $n$ vertices), only require $O(\log(n))$ inequalities in their compressed form \cite{kaibel2010constructing}.

This intuition is formalized as follows:

\begin{definition}[Extended Formulation]
  Let $P$ be a $d$-dimensional polytope, $Q$ an $m$\nobreakdash-dimensional polytope and $\pi:\R^m \to \R^d$ a linear projection.
  The pair $(Q,\pi)$ is called an \emph{extended formulation} of $P$, if $\pi(Q)=P$. The \emph{size} of $(Q,\pi)$ is defined as the number of facets of $Q$.
\end{definition}

\begin{definition}[Extension Complexity]
  The \emph{extension complexity} of a polytope $P$ is the smallest possible size of an extended formulation of $P$. It is usually denoted with $\xc(P)$.
\end{definition}

Polytopes with a small extension complexity can be ``compressed'' very well with extended formulations.
Such simplified formulations can be used for building faster algorithms for hard linear programs \cite{yannakakis1991expressing}, which seems to be a driving reason for the popularity of research in extended formulations.


\subsection{Overview of Current Research}

The application of extended formulations for solving hard linear problems seems promising. However, research in the past years shows unpromising results:

\begin{itemize}
  \item The \emph{traveling salesman polytope} can't have polynomial extension complexity \cite{fiorini2015exponential}. For this problem, extended formulations don't provide an improvement over traditional methods.
  \item The results for the \emph{matching polytope} are even worse: In contrast to the polynomial-time algorithm for solving this problem \cite{ford1956maximal}, there is no polynomial-size extended formulation \cite{rothvoss2017matching}.
\end{itemize}

That's why the study of polygons has importance as a benchmark for extended formulations in general (\citeauthor{braun2015matching} called it ``prototypical importance'' \cite{braun2015matching}).
Current research focuses on finding good bounds for the extension complexity, rather than developing algorithms for constructing extended formulations.

There are currently two main approaches for finding bounds for the extension complexity: The first one is the natural geometric approach. The second one is a linear-algebraic approach based on a remarkable result of \textcite{yannakakis1991expressing}, which we recall briefly:

\begin{definition}[Slack Matrix]
  Let $P$ be a polytope.
  Then a \emph{slack matrix}\footnote{Note that the slack matrix depends on the representation of $P$. So there is no unique slack matrix to a given polytope.} $M$ of $P$ is a nonnegative matrix, whose rows are indexed by the vertices of $P$ and whose columns are indexed by the linear constraints of some representation of $P=\{x \in \R^d \mid Ax \leq b\}$.
  The entries of $M$ are defined as $m_{ij} = b_j - a_j v_i$, where $a_j$ is a row of $A$ and $v_i$ is a vertex of $P$.
\end{definition}

\begin{definition}[Nonnegative Rank]
  The \emph{nonnegative rank} of a nonnegative matrix ${M \in \R_{\geq 0}^{n \times m}}$ is the smallest integer $k$, for which $M = TU$, where $T \in \R_{\geq 0}^{n \times k}$ and $U \in \R_{\geq 0}^{k \times m}$ are nonnegative matrices.
  We define $\rank_+(M) := k$.
\end{definition}

The nonnegative rank of $M$ can equivalently be defined as the minimum $r$ for which the matrix~$M$ can be written as the sum of $r$ nonnegative matrices of (ordinary) rank $1$.

\begin{theorem}[see \cite{yannakakis1991expressing}]
  Let $P$ be a polytope and $M$ a slack matrix of $P$. Then $$\xc(P) = \rank_+(M) .$$
\end{theorem}

Because this document is focused on polygons, we give a concise overview of the history of bounds and conjectures for the extension complexity of those:

Define $\P_n$ as the set of all convex polygons with $n$ vertices. Then
$$\pc: \N \to \N,\ n \mapsto \max \{\xc(P) \mid P \in \P_n\}$$
is called \emph{polygon complexity}. It describes the largest extension complexity for a polygon with $n$ vertices.

The lower bound $\pc(n) \in \Omega(n^{1/2})$ is quite easy to prove, which was done many times, for example by \textcite{fiorini2012extended} using a counting argument. In this paper we will provide another proof for this lower bound.

The upper bound is more challenging. Only few improvements have been made for a long time and the trivial upper bound $\pc(n) \leq n$ could only be improved by constant factors. One such improvement was done independently by \textcite{shitov2014upper} and \textcite{padrol2014polygons} and proved that $\pc(n) \leq (6n+6)/7$. The basis for their proofs was $\pc(7)=6$, i.e.\ each heptagon has extension complexity at most 6.

Because of missing improvements, $\pc(n) \in \Theta(n)$ was conjectured \cite{fiorini2012extended}. This was refuted by \textcite{shitov2014sublinear} proving $\pc(n) \in o(n)$. He later improved the upper bound to $\pc(n) \in O(n^{2/3})$ \cite{shitov2020sublinear}, which is the best known bound so far.

In summary, this is currently known about the polygon complexity:
$$\pc(n) \in \Omega(n^{1/2}) \cap O(n^{2/3})$$

There is another recent paper by \textcite{kwan2020extension}, which focuses on cyclic polygons\footnote{A polygon is called cyclic, if all its vertices lie on a circle.}. The authors proved that $\xc(P) \in O(n^{1/2})$ for all cyclic $n$-gons.
This led them to propose that this upper bound could also hold for general $n$-gons since ``cyclic polygons seem to represent quite a diverse cross-section of the space of all polygons'' \cite[p.\,3]{kwan2020extension}.



\subsection{Overview of this Document}

In this paper we focus on the key ideas behind the discussed papers \citetitle*{shitov2020sublinear}, \textcite{shitov2020sublinear} and \citetitle*{kwan2020extension}, \textcite{kwan2020extension}. We will make use of examples and skip most technical detail.

In the first part we examine the results which led to the best known upper bounds.
We will start with \textcite{shitov2020sublinear}, which provides the upper bound of $O(n^{2/3})$ for arbitrary $n$-gons by a purely geometric approach.
From there we have a look at cyclic $n$-gons and the upper bound of $O(n^{1/2})$, which was provided by a semi-geometric/semi-algebraic approach by \textcite{kwan2020extension}.
Next, we try to compare these two approaches, as far as they are comparable. First on a high level comparing key ideas and later on a more technical level.

In the second part we give another proof of the lower bound and present the applied theorem \cite[Theorem~1]{averkov2016maximum}. We also give a quick overview of the key ideas behind it.
