\section{Introduction} 
% todo March 10
% todo: citations

\subsection{Extension Complexity}

To describe a regular hexagon with inequalities, 6 inequalities are required. However by projecting a 3-dimensional polytope, one can achive the same polygon with 5 inequalities, see \ref{fig:hexagon}. 

\begin{figure}[h]
  \centering
  \includegraphics[scale=0.5,trim={0 2.5cm 0 2cm},clip]{assets/hexagon}
  \caption{A regular hexagon as a projection of a polytope with 5 factes (see \cite{kwan2020extension}).}
  \label{fig:hexagon}
\end{figure}

This is the key idea behind extended formulations. Simplifing the representation of polygons by projecting simpler (higher-dimensional) polytopes "down" to the desired polytope.
One can think of this operation as "compressing" the representation. This illustration is inspired by the fact, that regular $n$-gons (i.e. polygons with $n$ vertices), only require $O(log(n))$ inequalities in their compressed form, since their very nature is regular.

This intuition is formalized like this:

\begin{definition}[Extended Formulation]
  Let $P$ be a $d$-dimensional polytope\footnote{A polytope is the convex hull of a finite set of points}, $Q$ a $m$-dimensional polytope and $\pi:\mathbb{R}^d \to \mathbb{R}^m$ a linear projection.
  The pair $(Q,\pi)$ is called an \textit{extended formulation} of $P$, if $\pi(Q)=P$. The \textit{size} of $(Q,\pi)$ is defined as the number of factes of $Q$.
\end{definition}

\begin{definition}[Extension Complexity]
  The \textit{extension complexity} of a polytope $P$ is the smallest possible size of an extended formulation of $P$. Is is usually denoted with $xc(P)$.
\end{definition}

Polytopes with a small extension complexity can be "compressed" very well with the approach of extended formulations.
Such simplified formulations can be used for improving linear programs, since optimization is typically faster, the less inequalities are required.

\subsection{Overview of Current Research}
The application of extended formulations for solving hard linear problems seems promising. But reserach in the past years shows some bad examples:
\begin{itemize}
  \item It was shown, that the \textit{traveling salesman} polytope can't have polynomial extension complexity. So here extended formulations don't provide an improvement over traditional methods.
  \item The results for the \textit{matching} polytope are even worse: In contrast to the polynomial-time algorithm for solving this problem, there is no polynomial-size extended formulation.
\end{itemize}
So the study of polyons has importance as a benchmark for extened formulations in general.
That's why current research focuses on finding good bounds for the extension complexity, rather than developing algorithms for constructing such extensions.

% todo: slack matrix

Because this document is focused on polygons, we recall briefly recall the history of bounds and conjectures for the extension complexity:

Define $\mathcal{P}_n$ as the set of all convex polygons with $n$ vertices. Then $$pc:\mathbb{N} \to \mathbb{N}, pc(n) = \max_{P \in \mathcal{P}_n} xc(P)$$ is called the \textit{polygon complexity}, i.e. the largest extension complexity for a polygon of size~$n$.

The lower bound $pc(n) \in \Omega(n^{1/2})$ is quite easy to prove, which was done by the authors of @todo by a counting argument. In this paper we will provide another approach to proving this lower bound. 

But the upper bound is more challenging. Only few improvements were made for a long time and the trivial upper bound $pc(n) \leq n$ could only be improved by constant factors. One such improvement was done independently by Shitov and Padrol \& Pfeifle and proved that $pc(n) \leq (6n+6)/7$. The basis for their proved was the fact that $pc(7)=6$, i.e. each heptagon has extension complexity at most 6.
It was conjectured, that $pc(n) \in \Theta(n)$, which was falsified by Shitov by proving $pc(n) \in o(n)$. He later improved the upper bound to $pc(n) \in O(n^{2/3})$, which is the best known bound so far.

So this is currently known about the polgon complexity:
$$pc(n) \in \Omega(n^{1/2}) \cup O(n^{2/3})$$

There is another recent paper, which focuses on cyclic polygons (all vertices lie on a circle). The authors proved that $xc(P) \in O(n^{1/2})$ for all circular $n$-gons. 
This led the authors to propose, that this upper bound could also hold for general $n$-gons, since "cyclic polygons seem to represent quite a diverse cross-section of the space of all polygons".

\subsection{Overview of this Document}
\begin{itemize}
  \item structure, style: Focus on examples and key ideas, refer to original for techincal details
  \item upper bound of $O(n^{2/3})$ with geometric approach
  \item upper bound of $O(n^{1/2})$ for circular n-gons with semi-geometric / semi-algebraic approach
  \item compare, how both approaches differ
  \item lower bound (multiple proof already done)
\end{itemize}
