\section*{Abstract}
% todo @ March 21

{\large
  Anforderungen aus E-Mail
}

In der [zweiten Arbeit](https://arxiv.org/pdf/2006.08836.pdf) geht es auch um höhere Dimensionen. Wenn ich es richtig verstehe, zeigt Shitov, dass jedes n-gon eine erweiterte Formulierung der Größe $O(n^{2/3})$ hat. In zweiteren Paper wird gezeigt, dass man diese Schranke zu $O(n^{1/2})$ verbessern kann, wenn alle Ecken des Polygons auf einem Kreis liegen.

Wie wäre es, wenn Sie in Ihrer Arbeit die \textbf{Grundidee von Shitovs Argumenten} herausarbeiten und dann darlegen, wo sich \textbf{die Wege trennen} um zu $O(n^{2/3})$ bzw. $O(n^{1/2})$ zu gelangen. Zudem wäre es schön, wenn Sie die \textbf{untere Schranke} von $\Omega(n^{1/2})$ in Ihrer Arbeit darstellen. Letzteres funktioniert mit Methoden, die Sie aus Ihrem Seminarvortrag kennen. Diese Aspekte mal alle in einer Arbeit zu haben, würde ich schön finden. Beim Umfang einer Bachelorarbeit wird es Ihnen nicht möglich sein, alle Argumente aus der Shitov-Arbeit im Detail darzustellen. Das ist auch nicht nötig. Viel besser wäre es, die \textbf{grundlegenden Aussagen zu sammeln und zu erklären, wie sie schließlich zu den Aussagen führen}. Für welche der Aussagen Sie dann die \textbf{Beweise noch einmal vorführen}, ist Ihnen überlassen.
