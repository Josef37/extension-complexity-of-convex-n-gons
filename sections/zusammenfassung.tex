\section*{Zusammenfassung}

Das Thema dieser Arbeit ist die \emph{Extension Complexity} konvexer Polygone, was auf Deutsch so viel wie \emph{Erweiterungskomplexität} bedeutet. Die Idee dahinter ist, ein Polytop als lineare Abbildung eines höherdimensionalen Polytops darzustellen. Diese \emph{erweiterte Darstellung} nennt man \emph{Extended Formulation} und dessen \emph{Größe} ist die Anzahl der Facetten des höherdimensionalen Polytops.

Die \emph{Erweiterungskomplexität} eines Polytops $P$ ist nun die kleinste Größe aller erweiterten Darstellungen dieses Polytops und man bezeichnet diese mit $\xc(P)$.

Wir behandeln in dieser Arbeit die besten bekannten Schranken dieser Erweiterungskomplexität $\xc(P)$ für konvexe Polygone.

Bisher ist bekannt, dass für ein konvexes Polygon $P$ mit $n$ Ecken, hier als $n$-Eck bezeichnet, $$\xc(P) \in \Omega(n^{1/2}) \cap O(n^{2/3})$$ gilt. Liegen alle Punkte von $P$ auf einem gemeinsamen Kreis, ist $P$ also \emph{zyklisch}, lässt sich die Abschätzung asymptotisch präzise zu $$\xc(P) \in \Theta(n^{1/2})$$ verbessern.

Wir geben in dieser Arbeit einen Überblick über die Beweise, die zu den oben genannten Schranken führen, und konzentrieren uns vor allem auf Zusammenhänge der Aussagen und Grenzen der Vorgehensweisen.

Wir beginnen damit, den Beweis für $\xc(P) \in O(n^{2/3})$ für allgemeine $n$-Ecke $P$ darzustellen und dessen Grenzen aufzuzeigen.\\
Das Vorgehen ist dort rein geometrischer Natur und versucht für jedes $n$-Eck eine Teilfolge~$u$ von Ecken zu finden, die groß genug ist und eine möglichst kleine Erweiterungskomplexität besitzt (für $m \in \Omega(n^{2/3})$ Ecken $\xc(u) \in O(m^{1/2})$). Durch diese Aussage kann man dann induktiv zeigen, dass $\xc(P) \in O(n^{2/3})$ gilt.\\
Ein zentraler Satz in diesem Vorgehen erlaubt uns einfache, dreidimensionale erweiterte Darstellungen eines vereinfachten Polygons zusammenzufügen, um eine erweiterte Darstellung unseres gewünschten Polygons zu erhalten. Wir zeigen, dass dieser Satz für ein $n$-Eck nur erweiterte Darstellungen der Größe $\Omega(n^{1/2})$ erzeugen kann.

Als nächstes geben wir einen Überblick zum Beweis von $\xc(P) \in O(n^{1/2})$ für zyklische $n$-Ecke $P$, wobei diese obere Schranke asymptotisch optimal ist.\\
Das Vorgehen benutzt hier im Kern den linear-algebraischen Ansatz, der Eigenschaften gewisser Matrizen mit der Erweiterungskomplexität eines Polygons verbindet.

Anschließend vergleichen wir beide Vorgehensweisen und versuchen Ähnlichkeiten und Unterschiede auszuarbeiten, auch wenn die beiden Ansätze grundlegend verschieden sind. Wir wiederholen auch in asymptotischer Betrachtung, wie die beiden Ansätze die Schranken hergeleitet haben.

Im letzten Abschnitt geben wir noch einen weiteren Beweis für $\xc(P) \in \Omega(n^{1/2})$ für zyklische $n$-Ecke, damit auch für allgemeine $n$-Ecke. Wir verwenden dafür einen Satz, der uns erlaubt diese untere Schranke für eine Familie von Polygonen abhängig von ihrem gegenseitigen Abstand zu formulieren.

Wir schließen die Arbeit mit einer Übersicht über aktuelle Vermutungen zur Erweiterungskomplexität von Polygonen.
