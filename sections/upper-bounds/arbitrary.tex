\subsection{Arbitrary Convex n-gons}
% todo @ March 13 to 15
% todo: choose right fraction (\frac or "/") to look nice (ctrl+F)
% todo: add citations from original paper (to every definition, theorem, ...)
% todo: remove some statements based on how much of chapter 9 to 11 is used
% todo: make sure all "cw" and "ccw" are replaced with full words.

In this section we give a detailed overview of \cite{shitov2020sublinear}, which proofs $\pc(n) \in O(n^{2/3})$ for arbitrary $n$-gons. 
We focus on the underlying ideas and present only important arguments in more detail. In contrast to the source, we start the proof "backwards", presenting  the results first, since in doing so we can highlight the reasoning more clearly.

The gist of the approach is this: 
\begin{enumerate}
  \item We cut the polygon into smaller slices, for which we can provide small extended formulations. Joining these together preserves this property.
  \item We prove this small extended formulation by iteratively finding a large enough subset of vertices in the slice, for which we can build a small extended formulation. Joining these subsets together will also preserve this property.
  \item We build the small extended formulation by building 3-dimensional extended formulations for a surrounding polygon. These are "glued" together resulting in an extended formulation for the original set of vertices.
\end{enumerate}



\subsubsection{Main Result}

\begin{theorem}\label{theorem:xc}
  Every convex $n$-gon $P$ has $\xc(P) \leq 147 n^{\frac{2}{3}}$.
\end{theorem}

Naturally, we need to gather more statements, before we can prove this theorem. 

The starting point is the following lemma, adpopted from \cite[Proposition 3.1.1]{weltge2015sizes}, which states that $\xc$ behaves well for unions of polytopes:
\begin{lemma}\label{lemma:union}
  Let $P$ and $Q$ be polytopes in $\R^d$ -- each different from a point \footnote{In regards to polygons, one is tempted to think of these polytopes as being disjunct. But they can be arranged arbitrarily}. Then $$\xc(\conv(P \cup Q)) \leq \xc(P) + \xc(Q) .$$
\end{lemma}

This lemma allows us to consider a partition of the polygon, since we can merge the extended formulations of those parts.

Now we will go on to define those smaller parts, which we can handle well.

\begin{definition}[Turning angle]
  If $P\subset\R^2$ is a polygon, then the \textit{turning angle} at a vertex $v$ is $\pi-\angle v_-vv_+$, where $v_-$ and $v_+$ are two vertices adjacent to $v$ in $P$. In other words, it is the amount the angle at $v$ deviates from $\pi$. The \textit{turning angle} of an edge $e$ of a polygon is the sum of the turning angles at the two endpoints of $e$.
\end{definition}

\begin{definition}[Correct sequence]
  A sequence $v=(v_1,\ldots,v_n)$ of distinct points on a plane is called \textit{correct} if these points are the vertices of their convex hull $P$ and the segment between any pair of consecutive points in $v$ is an edge of $P$.
\end{definition}

\begin{definition}[Thin sequence]
  Let $\alpha\in(0,\pi)$ and $n \geq 3$. A sequence $v=(v_1,\ldots,v_n)$ is called $\alpha$\textit{-thin} if $v$ is correct and, additionally, the turning angle of the edge $\conv\{v_1, v_n\}$ in the polygon $\conv v$ is greater than $2\pi-\alpha$, that is, $$\angle v_n v_1 v_2+\angle v_1v_nv_{n-1}<\alpha.$$ We say that $v$ is \textit{thin} if it is $\alpha$-thin for some $\alpha\in(0,\pi)$.
\end{definition}
% todo?: motivation for meeting on the same side

\begin{observation}[Splitting into thin sequences]\label{observation:splitting}
  Let $P$ be a convex polygon with $n$ vertices, and let $q \geq 3$ be an integer. Then the vertices of $P$ can be partitioned into at most $q$ sets each of which is either a point or a pair of points, or else a set that forms a $2\pi/q$-thin sequence.
\end{observation}
% todo: image picpic11.png
% todo?: proof or gist: Collect vertices until you can no more (by thinness). Then start new set.

Based on Observation~\ref{observation:splitting} we choose to split the original $n$-gon into $12$ distinct $\pi/6$-thin sequences. Joining those (constantly many) sequences will not change the asymptotical bound for the extension complexity of the original polygon in light of Lemma~\ref{lemma:union}.

Now we will go on to show that each $pi/6$-thin sequence has extension complexity $O(n^{2/3})$. We do this by iteratively extracting a subsequence with a small extended formulation.

\begin{theorem}\label{theorem:subsequence}
  Let $v$ be a $\pi/6$-thin sequence with $n = 1024\tau^3 + 8\tau$ vertices, where $\tau \in \N$. 
  Then $v$ contains a subsequence $u$ with $\abs{u} \geq 4\tau^2$ and $\xc(u) \leq 12\tau$.
\end{theorem}

Proving this theorem is the objective of the rest of the paper. We preempted it to provide a better understanding of the results.

% todo: explain motivation/idea
\begin{corollary}\label{corollary:subsequence}
  Let $v$ be a $\pi/6$-thin sequence with $n > 263\,000$ vertices. 
  Then $v$ contains a subsequence $u$ with $\abs{u} \geq \frac{1}{36} n^{\frac{2}{3}}$ and $\xc(s) \leq \left( \frac{72}{43}n \right)^{\frac{1}{3}}$.
\end{corollary}

\begin{proof}[Proof outline]
  Apply Theorem~\ref{theorem:subsequence} with $\tau=\left\lfloor (n/1032)^{1/3} \right\rfloor$, which gives the desired result for $n > 263\,000$.
\end{proof}

% todo: explain idea (induction)
\begin{corollary}\label{corollary:thin-xc}
  Let $v$ be a $\pi/6$-thin sequence. Then $\xc(v) \leq \frac{324}{\sqrt[3]{129}} n^{\frac{2}{3}}$.
\end{corollary}

% todo?: describe induction step in more detail
\begin{proof}[Proof outline]
  Use induction for $n > 263\,000$ and apply Corollary~\ref{corollary:subsequence} and Lemma~\ref{lemma:union} for the induction step.
\end{proof}

We can now prove the main result by joining the extended formulations for all $\pi/6$-thin sequences.

\begin{proof}[Proof outline of \ref{theorem:xc}]
  Using Observation~\ref{observation:splitting} we spilt the $n$-gon $P$ into twelve disjoint $\pi/6$-thin sequences\footnote{Technically we have to consider the case, when a sequence has less than three points. We omit it, since it does not provide further insight.} with sizes $n_1,\dots,n_{12}$.

  We apply Lemma~\ref{lemma:union} and Corollary~\ref{corollary:thin-xc} and get $$\xc(P)\leq\frac{324}{\sqrt[3]{129}}\left(\sum\limits_{i=1}^{12}n_i^{2/3}\right)\leq\frac{324}{\sqrt[3]{129}}\left(12\left(\frac{n}{12}\right)^{2/3}\right) < 147 n^{2/3} .$$
\end{proof}



\subsubsection{Building Small Extended Formulations}

Before we can determine the subsequences in Theorem~\ref{theorem:subsequence}, we need to find a way to build those extended formulations.

The key idea here is to omit some vertices to obtain a simpler surrounding polygon. For this polygon we build extended formulations with respect to the omitted vertices. If we now "glue" those extended formulations together in a special way, we get an extended formulation for the original polygon with small extension complexity.

Before we can formulate this theorem, we have to introduce \emph{acute polyhedra} and \emph{acute diagrams}. These help us build the extensions for the (simple) surrounding polygon.

\begin{definition}[Acute Polyhedron]
  Assume $P \subset \R^3$ is a polyhedron and $B$ is one of its factes. If all other facets $F \neq B$ of $P$ 
  share an edge with $B$ and
  the angle\footnote{The angle between two faces $A$ and $B$ with a common edge $e$ is defined as the angle between two oriented segments that lie on $A$ and $B$ respectively, have their origin on $e$ and are orthogonal to $e$.} between $B$ and $F$ is acute,
  then $P$ is called an \emph{acute polyhedron} with \emph{base} $B$.
\end{definition}
% todo: image of actue polyhedron and not acute polyhedron

\begin{definition}[Main edge]
  An edge $e$ of an acute polyhedron with base $B$ is called \emph{main}, if exactly one endpoint of $e$ lies on $B$.
\end{definition}
% todo: image of main edges (merge with image before)

\begin{lemma}
  Any base vertex of an acute polyhedron belongs to a unique main edge.
\end{lemma}
% todo: summarize proof

Now follows an important lemma, which allows us to construct an acute polyhedron for a given base, where all main edges but one have fixed direction with respect to the base.
\begin{lemma}\label{lemma:acute-direction}
  Let $V$ be a polygon with vertices $v_1,\dots,v_n$ and let $y_1,\dots,y_{n-1}$ be a set of inner points of $V$.
  Then there is an acute polyhedron $P$ with base $V$ such that, for any $i \in \{1,\dots,n-1\}$, the image of the main edge passing from $v_i$ under the orthogonal projection of $P$ onto $V$ is collinear to $v_i \wedge y_i$ \footnote{We use $x \wedge y$ to describe the line joining two points $x$ and $y$. It can also the intersection of two lines $x$ and $y$.}.
\end{lemma}
% todo: describe proof (go around building, make sure it does not contradict in the end, that's why you have to have one y less)

\begin{observation}\label{observation:acute-projection}
  Let $P$ be an acute polyhedron with base $B$.
  The orthogonal projection $\pi$ of $P$ onto the plane containing $B$ maps the non-base points of an acute polyhedron injecively into interior of $B$.
\end{observation}

\begin{definition}[Acute diagram and lifting]
  Let $P$ be an acute polyhedron with base $B$ and $\pi$ like in Observation~\ref{observation:acute-projection}.
  The image of all edges of $P$ under $\pi$ gives us a diagram, which we call the \emph{acute diagram} of $P$ relative to base $B$. 
  We also say that $P$ is an \emph{acute lifting} of the corresponding diagram.
\end{definition}
% todo: add image of polyhedron, diagram and flow/orientation

We can now reformulate Lemma~\ref{lemma:acute-direction} for acute diagrams:

\begin{corollary}\label{corollary:acute-direction}
  Let $V$ be a polygon with vertices $v_1,\dots,v_n$ and let $y_1,\dots,y_{n-1}$ be a set of inner points of $V$.
  Then there is an acute diagram with base $V$ such that, for any $i \in \{1,\dots,n-1\}$, the main edge from $v_i$ lies on $v_i \wedge y_i$.
\end{corollary}

Since we have shown that every acute polyhedron has an acute diagram, we want to describe the properties of acute diagrams, which allow us to lift them into an acute polyhedron.

\begin{lemma}[Properties of acute diagrams]\label{lemma:diagram-properties}
  Let $\Delta$ be the acute diagram of an acute polyhedron $P$ with base $B$.
  Then $\Delta$ is a planar straight-line graph such that

  (o) the base of $\Delta$ is $B$,
  
  (i) every node of $\Delta$ has degree at least three,

  (ii) the non-base nodes of $\Delta$ lie in the interior of the base,

  (iii) every edge of the base is an arc of $\Delta$,

  (iv) every bounded face $F$ of $\Delta$ contains exactly one arc $e_F$ of the base,

  (v) if a non-base arc $e$ of $\Delta$ separates faces $F$, $G$, then $e, e_F, e_G$ are concurrent.
\end{lemma}

% todo: Explain flow?
% todo: Summarize reasoning for lifting

\begin{lemma}[Lifting acute diagrams]
  A planar straight-line graph $\Delta$ satisfying (i)-(v) as in Lemma~\ref{lemma:diagram-properties} is an acute diagram of some acute polyhedron $P$.
\end{lemma}
% todo?: repeat or summarize proof

We can now focus on the main result of the first part, which allows us to build small extended formulations by combining ("glueing") simple extensions for a surrounding polygon.

% todo: insert image example
\begin{theorem}[Glueing acute extensions together, {\cite[Theorem 28]{shitov2020sublinear}}]\label{theorem:glueing}
  Let $P$ be a polygon with vertex set $V$ and $\emptyset \neq S \subset V$ be a proper subset of all vertices. Fix $\delta \geq 1$. Assume

  (i) for any $s \in S$ there are two vertices $s'$, $s''$ on two edges of $P$ adjacent to $s$,

  (ii) there are $\delta$ points $\left\{s^1, \dots, s^\delta \right\}$ in the interior of the triangle ${T_s = \conv \left\{s,s',s''\right\}}$,
  
  (iii) the triangles $T_s$ are disjoint for different $s$ and

  (iv) for any $i \in \{1,\dots,\delta\}$ there exists and acute diagram $D^i$ with base face $P$ such that, for any $s \in S$, the segment between $s$ and $s^i$ is a subset of the main edge of $D^i$ passing from $s$. 
  
  $$\Rightarrow \xc\left( \conv\left( (V \setminus S) \cup \bigcup_{s \in S} \left\{ s',s'',s^1,\dots,s^\delta \right\}  \right) \right) \leq \abs{V} + \abs{S} + \delta$$
\end{theorem}
% todo: proof theorem 28

\begin{remark}\label{remark:pitfall}
  There is a subtle difference between Theorem~\ref{theorem:glueing} and Corollary~\ref{corollary:acute-direction}.

  In the theorem we assume that "the segment between $s$ and $s^i$ is a \textbf{subset} of the main edge", but in the corollary we construct a diagram for which "the main edge from $v_i$ lies on $v_i \wedge y_i$", implying that $y_i$ can lie outside the main edge. 

  So we can't just apply Theorem~\ref{theorem:glueing} to any choice of points fulfilling (i)-(iii).
\end{remark}

To understand Theorem~\ref{theorem:glueing} better, we will now look at the limits it has in application. This observation is aside from the main argument, so feel free to skip it.

\begin{observation}[Limits of Theorem~{\ref{theorem:glueing}}]\label{observation:limits-of-glueing}
  Let $P$ be a polygon with $n$ vertices. By applying Theorem~\ref{theorem:glueing} we obtain an extended fromulation for $P$ with size $\Omega(\sqrt{n})$.
\end{observation}
\begin{proof}
  We set $P = \conv\left( (V \setminus S) \cup \bigcup_{s \in S} \left\{ s',s'',s^1,\dots,s^\delta \right\} \right)$ as in Theorem~\ref{theorem:glueing}. We define $v := \abs{V}$ and $s := \abs{S}$. Throughout this proof we assume that we can apply Theorem~\ref{theorem:glueing} for our choice of $v$, $s$ and $\delta$. Then
  \begin{equation}\label{eq:limit-n}
    n \leq (v-s) + s(2+\delta) = v + s(1+\delta).
  \end{equation}

  With \eqref{eq:limit-n} we can express $\delta$ dependent on $n$, $v$ and $s$:
  \begin{equation*}
    \delta \geq \frac{n-v}{s} - 1
  \end{equation*}
  We now define $\delta := (n-v)/s$ as the minimal possible value for given $n$, $v$ and $s$, since the size of the extended formulation increases with $\delta$. This way we can define $f_n(v,s) := v + s + \delta = v + s + (n-v)/s$ as the size of the extended formulation for given $v$ and $s$.

  $\frac{\partial}{\partial v} f_n(v,s) = 1 - \frac{1}{s} > 0$ shows that $v$ should be chosen minimal for minimal extension size. So we set $v = s+1$ according to Theorem~\ref{theorem:glueing}.

  \begin{align*}
    f_n(s+1, s) &= (s+1) + s + \frac{n-(s+1)}{s} \\
    &= 2s + \frac{n-1}{s}\\
    \frac{\partial}{\partial s} f_n(s+1,s) &= 2 - \frac{n-1}{s^2} \Rightarrow s_{min} := \sqrt{\frac{1}{2}(n-1)} \\
    \frac{\partial^2}{\partial^2 s} f_n(s+1,s) &= \frac{2(n-1)}{s^3} > 0
  \end{align*}
  
   This shows that $f_n$ has a minimum at $s_{min}$ with $v_{min}=s_{min}+1$.

  Finally we can find the minimal value for $f_n$.
  \begin{align*}
    f_n(s_{min}+1, s_{min}) = 2\sqrt{2(n-1)}
  \end{align*}
\end{proof}



\subsubsection{Finding Good Subsequences}

After defining this central theorem, we have to make a way to apply it to general polygons (see Remark~\ref{remark:pitfall}, why we can't do this until now).
Therefore we formalize the notion of "good" sequences with respect to Theorem~\ref{theorem:glueing}.

\begin{definition}[G-envelope]
  % todo
\end{definition}
% todo: Image of G-envelope

\begin{definition}[G-good]
  % todo
\end{definition}

\begin{lemma}\label{lemma:glueing-reformulation}
  % todo?: Lemma 41? Maybe not needed (depends on chapters 9 to 11 formulation)
\end{lemma}

Now we have to find another way to determine which seqeunces are actually \emph{G-good}.
\begin{lemma}\label{lemma:ray-good}
  % todo: Lemma 42: ray definition of G-good
\end{lemma}
% todo: prove it?

We now need to define one more property of a sequence, by which we will split our upcoming considerations.

\begin{definition}[Slanted sequence]
  % todo
\end{definition}

Now we can split the search for a good subsequence into two cases:
\begin{enumerate}
  \item The original sequence has a large enough slanted subsequence.
  \item The original sequence does not have such a subsequence.
\end{enumerate}
We will show that for each case we can find a large enough subsequence, for which Theorem~\ref{theorem:glueing} can be applied well.

% todo: state Lemma 49
% todo: formulate list
Lemma 49
\begin{itemize}
  \item Pick $4\tau$ not ccw-slanted subsequence with $m$ vertices, which we'll index by $q$.
  \item For each subseq. find the points $i, \hat{\imath}, j, \hat{\jmath}$ and the set $\Delta$ of $2\delta$ $k$'s which break the slanted definition.
  \item Assume w.l.o.g. that $\omega_q := ...$ lies left of half of each $\Delta_q$, since assumptions are symmetric for $v$.
  \item Pick $2\tau$ subseq.s with this property.
  \item Case 1: Half of those have at least half of the rays left
  \begin{itemize}
    \item Pick those $\tau$ subseq.s and for each the $\delta$ vertices with ray to the left.
    \item Build $G$ out of those $\delta$ vertices, so the ray in Lemma~48 is left and $\omega$ is right (by defining the point that preserves $\rho$ and $\omega$).
    \item By Lemma 48 it is $G$-good when stretched
    \item Apply Lemma 41 to this stretched seq.
    \item Cut this stretched seq. back with $\tau+1$ cuts
  \end{itemize}
  \item Case 2: Half of those have at least half of the rays not left
  \begin{itemize}
    \item Pick those $\tau$ subseq.s and for each the $\delta$ vertices with ray not to the left.
    \item The angle of those rays to the base is $>\alpha$, by $\beta \geq \alpha$, $\alpha$-thinness and unslantedness.
    \item Because is is not left and in an $\alpha$-thin sequence, is has to leave through the interior of the base.
    \item Lemma 42 (ray = good) shows it's $G$-good.
    \item Apply Lemma 41 for the bound.
    \item Add some arbitrary points to have a large enough subseq., which will influence $\xc$ at most linearily
  \end{itemize}
\end{itemize}

% todo: State Theorem 57 and explain the key idea behind it -> Trace Chapter 10 and 11 (draw sketches to all proofs for understanding them yourself), maybe only state Lemma 55, since it comes from direct computation, give an overview of Lemma 52
% todo: formulate
Theorem 57
\begin{itemize}
  \item Assumptions / Premises
  \begin{itemize}
    \item $n=8\delta^2$
    \item $\pi/6$-thin (apply Lemma 55)
    \item slanted angle $\pi/3$ (apply Lemma 55)
    \item tolerance $2\delta$ (Have $2\delta$ points for $G$ envelope)
  \end{itemize}
  \item Bulid a cw-decreasing seq. out of every $4\delta^{th}$ points
  \item Case 1: There are many points in this seq. -> Apply Corollary 52
  \item Case 2: There are few points in this seq.
  \begin{itemize}
    \item Look at all sequences of every $4\delta^{th}$ point you didn't take in constructing. They are \emph{perfect} by construction.
    \item Split them into length $8$ subseq. (also \emph{perfect}), from which there are at least $\frac{1}{4}\delta - h$
    \item Every such subseq. has $5$ points, where the ray leaves through the base (Obs. 45, Lemma 55). 
    \item Since you chose every $4\delta^{th}$ point, you can find another $4\delta - 2\delta - 1 = 2\delta - 1$ such rays.
    \item All those seq.s are good by Lemma 42
    \item Count vertices and apply Lemma 41 to get your estimate.
  \end{itemize}
\end{itemize}

Now we have all building blocks to tackle proving Theorem~\ref{theorem:subsequence}, which states that for a $\pi/6$-thin sequence $v$ with $n = 1024\tau^3 + 8\tau$ vertices it contains a subsequence $u$ with $\abs{u} \geq 4\tau^2$ and $\xc(u) \leq 12\tau$. In light of \ref{observation:limits-of-glueing} this subsequence makes optimal use of Theorem~\ref{theorem:glueing} in an asymptotical way.

\begin{proof}[Proof of Theorem~\ref{theorem:subsequence}]
  % todo: Prove or just give an overview of the proof
\end{proof}



\subsubsection{Conclusion and Summary}

% todo: summarize everything again briefly: split into thin -> inductively extract "good" subsequences -> slanted or not, you can glue simple extensions together
% todo?: possible improvements, applications? -> Use Theorem 28 otherwise? How to improve the bound by finding bigger subsequences (Theorem 28 is maxed out).









\newpage
{\Huge \textbf{Notes}}

Proofs worth repeating
\begin{itemize}
  \item Theorem 28
  \item Lemma 42 (good = ray)?
  \item Lemma 48?
  \item Lemma 25 / Corollary 26?
\end{itemize}

Examples worth making/drawing
\begin{itemize}
  \item all examples from source
  \item Examples of acute polytope/diagram (main edges + orientation in most proofs)
  \item Examples of thin sequences (one thin, one not thin and one thin with one side > 90 deg)
  \item Ray $\rho$
  \item G-envelope + G-good
  \item Slanted sequences: angle $\beta$
  \item Decreasing sequence?
  \item Perfect sequence
  \item Proof of Lemma 49, Case 1
\end{itemize}

\begin{tabular}{| p{50mm} | p{100mm} |}
  \hline
  \multicolumn{2}{|c|}{\textbf{Main results per chapter}}\\ \hline
  3. Preliminaries & Extension Complexity for union of polytopes \\ \hline
  4. Acute polyhedra & There is an acute polyhedron for a polygon, where all main edges except one have fixed direction \\ \hline
  5. Acute diagrams & There is an acute diagram for a polygon, where all main edges except one have fixed direction \\ \hline
  6. Glueing acute extensions together & Skip vertices of a polygon by joining acute diagrams to get an extended formulation which represents the original polygon \\ \hline
  7. Thin sequences & Splitting polygons is now a thing \& something about rays passing a horizontal line \\ \hline
  8. Good sequences & good sequences let you apply Theorem 28, identify good sequence via ray passing through top \\ \hline
  9. Extracting a slanted subsequence & When it's not slanted, it'll extend well (Lemma 49) \\ \hline
  10. Decreasing slanted sequences & Get estimates of xc for slanted sequences \\ \hline
  11. Extensions for slanted sequences & Subsequence of slanted sequence with good xc \\ \hline
  12. Completing the proof & Split the polygon, iteratively extract a subsequence and bound xc (see chapter 11 for slanted, chapter 9 for not slanted) \\ \hline
\end{tabular}
