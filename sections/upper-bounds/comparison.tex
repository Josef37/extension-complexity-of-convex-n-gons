\subsection{Comparison}
% todo: formulate and cite claims
\textbf{This subsection is not finished yet.}

In this section we compare the results and approach from \cite{shitov2020sublinear} and \cite{kwan2020extension} by working out general similarities or differences. We further look at how both bounds are achieved numerically (in an asymptotic view). In the end we examine how the proofs could be altered to use them in the other's setting.

We start with a quick overview of both results:

\begin{itemize}
  \item \cite{shitov2020sublinear} proves Theorem~\ref{theorem:xc}, which states that every $n$-gon $P$ has $\xc(P) \in O(n^{2/3})$.
  \item \cite{kwan2020extension} proves Theorem~\ref{theorem:cyclic-xc}, which states that every cyclic $n$-gon $P$ has $\xc(P) \in O(n^{1/2})$.
\end{itemize}

First we list the general similarities of both approaches:

\begin{itemize}
  \item Both follow the strategy of splitting the polygon into smaller slices.\\
  \cite{shitov2020sublinear} spilts the original polygon into $12$ slices, each $\pi/6$-thin, which are treated separately.\\
  \cite{kwan2020extension} splits the polygon into $O(n^{1/2})$ arcs containing $O(n^{1/2})$ vertices. But here they are treated as a whole.

  \item There is some notion of enclosing vertices in both.\\
  \cite{shitov2020sublinear} requires an envelope around vertices in its central theorem \ref{theorem:glueing}. It is used to build an extended formulation for this specific polygon.\\
  \cite{kwan2020extension} uses a "lampshade argument", which builds a polygon around vertices away from a set of facets, to show that a slice of the slack matrix has constant nonnegative rank.
\end{itemize}

Even though some ideas are common to both methods, they are more different than similar:

\begin{itemize}
  \item One key difference is the type of approach:\\
  \cite{shitov2020sublinear} uses a purely geometric method.\\
  \cite{kwan2020extension} uses the linear-algebraic strategy using slack matrices for its main reasoning. But lemmata are often proved geometrically. The approach has the advantage that transformation on the slack matrix, which don't alter the nonnegative rank, can not always be represented geometrically.

  \item There is another difference in how they treat subsets of vertices:\\
  \cite{shitov2020sublinear} iteratively extracts a large subsequence with small extension complexity (ignoring all other vertices).\\
  \cite{kwan2020extension} splits its consideration by constantly many colors, but always handles the dependencies to either all other vertices or all other facets. 
\end{itemize}

We go on analyzing numerically how each approach ends up with its asymptotical bound.

$\xc(P) \in O(n^{2/3})$ for arbitrary $n$-gons $P$ in \cite{shitov2020sublinear}:
\begin{itemize}
  \item subsequence $u$ with $m \in O(n^{2/3})$ vertices;
  \item extension complexity $\xc(u) \in O(m^{1/2}) = O(n^{1/3})$;
  \item iteration results in $\xc(P) \in O(n^{2/3})$
\end{itemize}

$\xc(P) \in O(n^{1/2})$ for cyclic $n$-gons $P$ in \cite{kwan2020extension}:
\begin{itemize}
  \item number of colors in $O(1)$;
  \item $\abs{\X} \in O(n^{1/2})$ arcs;
  \item for each color $\rank_+ K \in O(\abs{\X_c})$;
  \item each color column in slack has $\rank_+ M[V,F_c] \in O(\abs{\X_c} + n^{1/2})$;
  \item putting together $\rank_+ M \in O(\abs{\X} + n^{1/2}) = O(n^{1/2})$
\end{itemize}

Finally we have a look at, how each procedure can be altered to be used in the other's setting:

\begin{itemize}
  \item How is the circle property used? Can the approach be generalised (replace circle distance with edge distance)?\\
  The circle ensures there are scaling factors $\alpha_v$, such that the local vertex has the smallest slack of any vertices of the same index and color
  \item Is there a way to adopt a proof to the other case? Try Shitov's work on cyclic polygons. And try the other way around (you may have to use thin slices, too).
\end{itemize}
