\subsection{Comparison}
% todo @ March 19 to 20

% \cite{shitov2020sublinear,kwan2020extension}

\begin{itemize}
  \item general similarities 
  \begin{itemize}
    \item splitting of polygons
    \item enclosing vertices
  \end{itemize}
  \item general differences 
  \begin{itemize}
    \item splitting by angle vs count
    \item iteratively finding subsequences vs considering the whole even when splitting by colors
    \item purely geometric vs main reasoning algebraic and lemmata geometric
  \end{itemize}
  \item How does one end up at $n^{1/2}$ and the other at $n^{2/3}$. Summarize the numerical steps (asymptotically).
  \begin{itemize}
    \item $n^{1/2}$\\ 
    number of colors in $O(1)$;\\
    there are $\X \in O(n^{1/2})$ arcs;\\
    for each color $\rank_+ K \in O(\abs{\X_c})$;\\
    each color column in slack has $\rank_+ M[V,F_c] \in O(\abs{\X_c} + n^{1/2})$;\\
    putting together $\rank_+ M \in O(\abs{\X} + n^{1/2}) = O(n^{1/2})$
    \item $n^{2/3}$\\
    subsequence $v$ with $m := \abs{v} \in O(n^{2/3})$;\\
    extension complexity $\xc(v) \in O(m^{1/2}) = O(n^{1/3})$;\\
    iteration results in $\xc(P) \in O(n^{2/3})$
  \end{itemize}
  \item How is the circle property used? Can the approach be generalised (replace circle distance with edge distance)?\\
  The circle ensures there are scaling factors $\alpha_v$, such that the local vertex has the smallest slack of any vertices of the same index and color
  \item Is there a way to adopt a proof to the other case? Try Shitov's work on cyclic polygons. And try the other way around (you may have to use thin slices, too).
  \item could the general case be $O(\sqrt{n})$, too? (current conjectures)
\end{itemize}
