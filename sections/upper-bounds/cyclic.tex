\subsection{Cyclic n-gons}

In this section we give an overview of \cite{kwan2020extension}, which proves -- among other things -- that $\xc(P) \in O(n^{1/2})$ for an $n$-gon $P$ with vertices on a common circle.

The approach in this paper differs form \cite{shitov2020sublinear}, since it uses the linear-algebraic method utilizing slack matrices. Nonetheless we will try to compare both in the next subsection.

One important insight is the ``lampshade argument'', which the authors called this way because of the geometric interpretation in three dimensions (see Figure~\ref{fig:lampshade}). In the two-dimensional case, the simplified form can be interpreted like this:

Given a polygon $P$, consider a set of consecutive facets $F'$ and the set of vertices $V'$, which are not endpoints of $F'$. Then if we consider the $V' \times F'$ submatrix $M[V',F']$ of a slack matrix $M$, we have $\rank_+ M[V',F'] = O(1)$.\\
The resaon for this is that we can enclose $V'$ inside a polygon $Q$, such that all vertices of $Q$ are on the ``positive slack'' side of $F'$ (meaning $Q$ and $P$ lie on the same side of every $f \in F'$). This polygon $Q$ can be thought of as a ``polyhedral lampshade'' and one can always build such a polygon $Q$ with constantly many vertices .\\
Since $Q$ encloses $V'$, every vertex $v \in V'$ is a convex combination of the vertices in $Q$.\\
For every vertex $q$ of $Q$ consider the vector $u_q \in \R^{\abs{F'}}$ of slacks for all $f \in F'$, which is positive by construction of $Q$.\\
And since the slack function is affine-linear, one can convex combine the slack vectors for all $v \in V'$ (in regards to $F'$) from those $u_q$.\\
In other words, every row of $M[V',F']$ is a convex combination of these constantly many vectors $u_q$, which shows $\rank_+ M[V',F'] = O(1)$.

\begin{figure}[h]
  \centering
  \includegraphics[scale=0.7,trim={2.5cm 2cm 2.5cm 4cm},clip]{separation}
  \includegraphics[scale=0.7,trim={0 2cm 0 4cm},clip]{shade}
  \caption{On the left, a small patch of facets near the north pole is far away from a collection of vertices. On the right, a ``polyhedral lampshade'' encloses all the vertices in our collection, and lies entirely on the ``positive slack'' side of each of the facets in our patch. \cite[Figure 2]{kwan2020extension}}
  \label{fig:lampshade}
\end{figure}

The authors of the paper developed this argument even for higher dimensions, since the other theorems provide results for higher-dimensional polytopes. For example, they prove in \cite[Theorem 1.1]{shitov2020sublinear} that a polytope $P \in \R^d$ as the convex hull of $n$ random points on the $(d-1)$-dimensional unit sphere has ``asymptotically almost surely'' extension complexity in $\Theta(n^{1/2})$.

Since we focus on polygons in this paper, we will only cover the separate, two-dimensional theorem, for whose derivation we will give an outline.

\begin{definition}
  A polygon is called \emph{cyclic}, if all its vertices lie on a common circle.
\end{definition}

\begin{theorem}[{\cite[Theorem 1.3]{kwan2020extension}}]\label{theorem:cyclic-xc}
  Let $P$ be a cyclic polygon with $n \geq 18^2$ vertices.\\
  Then $\xc(P) \leq 24\,n^{1/2}$.
\end{theorem}

The approach follows these steps:

\begin{enumerate}
  \item Split the circle into arcs, containing approximately $n^{1/2}$ facets each.
  \item Color these arcs with constantly many colors, such that two arcs of the same color are ``well-separated''.
  \item Build a nonnegative matrix deduced from the slack matrix by rescaling rows and adding approximately $n^{1/2}$ vectors, such that the entries for vertices and factes of the same arc are zero.
  \item Apply the ``lampshade argument'' (in a more general form) for the rest of this matrix and obtain the desired bound.
\end{enumerate}

We will now go through these steps in more detail.

Let $P$ be a cyclic polygon, $V$ its set of vertices and $F$ its set of facets.

We divide the circle into a set $\X$ of $\lceil n^{1/2} \rceil$ arcs, such that each arc spans from one vertex to another vertex (including both) and contains at most $\lceil n^{1/2} \rceil$ facets, i.e. facets with both endpoints in the arc, and therefore at most $N := \lceil n^{1/2} \rceil\ + 1$ vertices.

\begin{definition}[Well-separated]
  We say that two arcs $X,X' \in \X$ with arc lengths $\varepsilon$,~$\varepsilon'$ are \emph{well-separated}, if the arc-distance between any two points $x \in X$ and $x' \in X'$ is at least $5 \min\{\varepsilon, \varepsilon'\}$.
\end{definition}

\begin{lemma}\label{lemma:arc-not-separated}
  For every arc $X \in \X$, there are at most $13$ arcs $X' \in \X$, which are not well-separated from $X$ and at least as long as $X$.
\end{lemma}

\begin{proof}
  We count how many such arcs fit into the ``not well-separated space''.\\
  Let $\varepsilon$ be the length of $X$. If we put $5$ arcs of length at least $\varepsilon$ to each side of $X$, each other arc would have arc-distance at least $5\varepsilon$ to $X$. So the maximal number of not well-separated arcs with length at least $\varepsilon$ distinct to $X$ is $10$.\footnote{Our results in this proof differ from the source. But since this number only changes a constant in the bound, we will go on with the original numbers.} 
\end{proof}

\begin{lemma}
  We can color the arcs in $\X$ with at most 14 colors, such that arcs of the same color are well-separated.
\end{lemma}

\begin{proof}
  Order the arcs by decreasing length and color them in that order. For an arc $X$, there are -- by Lemma~\ref{lemma:arc-not-separated} -- at most $13$ arcs of at least the same length not well-separated. So just pick a color (from a set of $14$ colors), which wasn't assigned to those at most $13$ arcs and proceed until all arcs are colored.
\end{proof}

Now color the arcs in $\X$ with colors labelled by $c=1,\dots,14$. Define $\X_c$ as the set of arcs with color $c$.

We will now split the columns of the slack matrix M by color, i.e. into slices $M[V,F_c]$, where $F_c$ are all facets in arcs with color $c$.\\
Later we can combine the results like this: $\rank_+M \leq \sum_{c=1}^{14} \rank_+ M[V,F_c]$

For an arc $X \in \X$ define $F^X$ or $V^X$ as the facets or vertices inside that arc. We also call a vertex $v$ \emph{local} to a facet $f$, if $x \in V^X$ and $f \in F^X$, i.e. if they are in the same arc.

Further we enumerate all vertices for each arc of color $c$, that is for each $X \in \X_c$ we define a bijection $\phi_X: V^X \to \{1,\dots,\abs{V^X}\}$. We define $\phi: V_c \to \{1,\dots,N\}$ as the union of all $V_X$, which is well-defined, since the sets $V_X \in V_c$ are disjoint as they are well-separated.

\begin{lemma}\label{lemma:alphas}
  For all $v \in V_c$ there is $\alpha_v > 0$, such that for all $X \in \X_c$, $f \in F^X$, $w \in V^X$ with $\phi(v)=\phi(w)$ we have $\alpha_v M_{v,f} \geq \alpha_w M_{w,f}$.
\end{lemma}

In other words: For each vertex of color $c$ there is a scalar, such that for each facet the scaled slack of the local vertex is the smallest for all vertices of the same index and color.

\begin{proof}[Proof outline]
  We can split our considerations by $\phi$, that is we can find $\alpha_v > 0$ separately for each set of vertices $v \in V_c$ with the same index $\phi(v)$.

  Let's fix some index $i \in \{1,\dots,N\}$. Now order all arcs $X$ of color $c$ with at least $i$ vertices decreasing by arc-length as $X_1,\dots,X_m$. Label the vertex $v \in V^{X_j}$ corresponding to $X_j$ with index $i = \phi(v)$ with $v_j$.

  Look at the submatrix $M \left[ \{v_1, \dots, v_m\},F^{X_1} \cup \dots \cup F^{X_m} \right]$ of the slack matrix, which we will treat kind of like a square matrix with one entry being $M \left[ \{v_j\},F^{X_j} \right]$.

  For each $j \in \{1,\dots,m\}$ and $f \in (F^{X_1} \cup \dots \cup F^{X_m}) \setminus F^{X_j}$ we have $M_{v_j, f} > 0$, since $v_j$ cannot be a vertex of $f$, because all $X_k$ disjoint as they are well-separated.

  Further for each $j \in \{1,\dots,m\}$, $k \in \{1,\dots,j-1\}$, $f \in F^{X_j}$ and $g \in F^{X_1} \cup \dots \cup F^{X_{j-1}}$ we have $M_{v_j,f}M_{v_k,g} \leq M_{v_j,g}M_{v_k,f}$. \\
  To show this, first observe that $X_j$ is the shortest of the regarded arcs. If we denote its length with $\varepsilon$, it has arc-distance at least $5\varepsilon$ of every arc $X_k$, because they are well-separated and ordered by decreasing length.\\
  The gist of the proof is following. Since $M$ is representation dependent, reformulate it as a ratio of distances:
  $$\frac{d(v_j,f)}{d(v_k,f)} \leq \frac{d(v_j,g)}{d(v_k,g)}$$
  This can be done, since $M_{v_k,g} = 0$ verifies the statement and $M_{v_k,f} > 0$ holds, as the arcs are well-separated.\\
  The idea is now this: Because $v_j$ is local to $f$, but $v_k$ and $g$ are relatively far away (at least $5\varepsilon$ in arc-distance), $d(v_j,f)$ is much smaller than $d(v_k,f)$ and $d(v_j,g)$. $d(v_k,g)$ may be large, but then we can argue -- using the triangle inequality -- that then also one of $d(v_k,f)$ or $d(v_j,g)$ has to be large.\\
  Remark that this is the \emph{only place we use that $P$ is cyclic}.

  After those two properties of $M$ are proved, we can apply a technical lemma \cite[Lemma 10.3]{kwan2020extension}, which gives us the desired factors $\alpha_{v_1}, \dots, \alpha_{v_m}$, such that $\alpha_{v_i} M_{v_i,f} \leq \alpha_{v_j} M_{v_j,f}$ for all $f \in F^{X_i}$.
\end{proof}

For each $i \in \{1,\dots,N\}$ we define vectors $t^{(i)} \in \R^{\abs{F_c}}$. For each $X \in \X_c$, $f \in F_X$:
\begin{equation*}
  t_f^{(i)} = 
  \begin{cases}
    0 &\text{if } i > \abs{V^X}\\
    \alpha_v M_{v,f} &\text{else, with } v = \phi_X^{-1}(i)
  \end{cases}
\end{equation*}
So the vectors $t_f^{(i)}$ are the scaled slacks for the local vertices (regarding $f$) with index $i$.

Now we define the matrix $K \in \R^{\abs{V} \times \abs{F_c}}$, for which we will later apply the (general) ``lampshade argument''. For all $f \in F_c$ we define:
\begin{equation*}
  K_{v,f} = 
  \begin{cases}
    \alpha_v M_{v,f} - t_f^{(\phi(v))} & \text{if } v \in V_c\\
    M_{v,f} & \text{else}
  \end{cases}
\end{equation*}
In other words, we manipulate the rows of $K$ with vertices of color $c$, by scaling them, such that for vertices with the same index the slack of the local vertex is smallest, and then subtracting that smallest scaled slack.

The purpose of this manipulation is that for each $X \in \X_c$, any vertex $v \in V^X$ and any facet $f \in F^X$, we have $K_{v,f} = 0$. So all entries of $K[V^X, F^X]$ are zero, while $K$ is still nonnegative (that's why we had to scale all rows first).

\begin{lemma}\label{lemma:rank-of-K}
  The matrix $K$ has nonnegative entries and satisfies $\rank_+ K \leq 8\abs{\X_c}$.
\end{lemma}

For the proof we need the generalized ``lampshade argument'', how it was originally proved in \cite[Lemma 3.1]{shitov2014sublinear}. It was simplified for this use case to allow simpler notation.

\begin{lemma}\label{lemma:shitov-lampshade}
  Let $P$ be a polygon, and let $V$, $F$ and $M$ be defined like above. 
  
  Let $X \subseteq V$ be a set of consecutive vertices of $P$, and let $F' \subseteq F$ be the set of facets of $P$ with both endpoints in $X$.

  Let $M'$ be a matrix with rows indexed by $V\setminus X$ and columns indexed by $F'$, such that the following condition holds for each vertex $v\in V\setminus X$:
  
  There are real numbers $\alpha_v>0$, $\beta_v\geq 0$ and a vertex $x_v\in X$ such that\\
  $M_{v,f}'=\alpha_v M_{v,f}-\beta_v M_{x_v,f}$ for all $f\in F'$.

  Then, if all the entries of $M'$ are nonnegative, we have $\rank_+ M'\le 8$.
\end{lemma}

We will not prove this lemma here. But remark, that it does not require $P$ to be cyclic.

\begin{proof}[Proof outline of Lemma~\ref{lemma:rank-of-K}]
  We partition $K$ into submatrices $K[V,F^X]$ for $X \in \X_c$ and show that each of those $\abs{\X_c}$ slices has nonnegative rank at most $8$.

  We know $K[V^X,F^X] = 0$ by definition of $K$. So it suffices to consider $K[V \setminus V^X, F^X]$. Now there are three possibilities for rows of this submatrix:

  \begin{itemize}
    \item If $v \notin V_c$, then $K_{v,f} = M_{v_f}$ for all $f \in F^X$.
    \item If $v \in V_c$ and $\phi(v) > \abs{V^X}$, then $K_{v,f} = \alpha_v M_{v,f}$ for all $f \in F^X$, because $t_f^{(\phi(v))} = 0$.
    \item If $v \in V_c$ and $\phi(v) \leq \abs{V^X}$, then $K_{v,f} = \alpha_v M_{v,f} - \alpha_{x_v} M_{x_v,f}$ for all $f \in F^X$, where $x_v \in V^X$ with $\phi(x_v) = \phi(v)$ is the vertex local to $f$ with the same index as $v$.
  \end{itemize}

  From this we can see that $K[V \setminus V^X, F^X]$ is nonnegative by the choice of $\alpha_v$ in Lemma~\ref{lemma:alphas}.

  To show $\rank_+ K[V \setminus V^X, F^X] \leq 8$ we apply Lemma~\ref{lemma:shitov-lampshade} for each of the above cases with $V^X$ as our set of consecutive vertices.\\
  We have to choose $\alpha_v$, $\beta_v$ and $x_v$ for each vertex $v \in V \setminus V^X$.

  \begin{itemize}
    \item $v \notin V_c$: $\alpha_v = 1$, $\beta_v = 0$ and $x_v \in V^X$ arbitrary.
    \item $v \in V_c$ and $\phi(v) > \abs{V^X}$: $\alpha_v$ already defined, $\beta_v = 0$ and $x_v \in V^X$ arbitrary.
    \item $v \in V_c$ and $\phi(v) \leq \abs{V^X}$: $\alpha_v$ already defined, choose $x_v \in V^X$ with $\phi(x_v) = \phi(v)$ like above and set $\beta_v = \alpha_{x_v}$.
  \end{itemize}

  By Lemma~\ref{lemma:shitov-lampshade} and $K[V^X,F^X] = 0$ we can conclude: $$\rank_+ K[V,F^X] = \rank_+ K[V \setminus V^X,F^X] \leq 8$$
\end{proof}

So there are $8\,\abs{\X_c}$ nonnegative vectors, such that each row of $K$ can be written as nonnegative linear combination of them. So together with the $N$ vectors $t^{(i)}$ we can write every row of $M[V,F_c]$ as a nonnegative linear combination of $8\,\abs{\X_c} + N$ vectors. Therefore $\rank_+ M[V,F_c] \leq 8\,\abs{\X_c} + N$.

Now we can summarize: $$\rank_+ M \leq \sum_{c=1}^{14} \rank_+ M[V,F_c] \leq 8\abs{\X} + 14N \leq 22\,n^{1/2} + 36 \stackrel{(n \geq 18^2)}{\leq} 24\,n^{1/2}$$
