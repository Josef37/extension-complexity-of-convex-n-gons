\subsection{Cyclic n-gons}

In this section we give an overview of \cite{kwan2020extension}, which proves -- among other things -- that $\xc(P) \in O(n^{1/2})$ for an $n$-gon $P$, whose vertices lie on a circle.

The approach in this paper differs form \cite{shitov2020sublinear}, since it uses the linear-algebraic method utilizing slack matrices. Nonetheless we will try to compare both in the next subsection.

One important insight is the "lampshade" argument, which the authors called this way because of the geometric interpretation. In the two-dimensional case, the simplified form can be interpreted like this:
% todo: explain lampshade (arc with facets, seperated vertices, lampshade in the positive slack, slack vectors, convex combinations)

The authors of the paper developed this argument even for higher dimensions, since the other theorems provide results for higher-dimensional polytopes. For example, they prove that a polytope $P \in \R^d$ as the convex hull of $n$ random points on the $(d-1)$-dimensional unit sphere has "asymptotically almost surely" extension complexity in $\Theta(n^{1/2})$.

Since we focus on polygons in this paper, we will only cover the seperate, two-dimensional theorem, for whose derivation we will give an outline.

\begin{definition}
  A polygon is called \emph{cyclic}, if all its vertices lie on a common circle.
\end{definition}

\begin{theorem}[{\cite[Theorem 1.3]{kwan2020extension}}]
  Let $P$ be a cyclic polygon with $n$ vertices.\\
  Then $\xc(P) \leq 24\,n^{1/2}$.
\end{theorem}

The approach follows these steps:

\begin{enumerate}
  \item Split the circle into arcs, containing approximately $n^{1/2}$ facets each.
  \item Color these arcs with 14 distinct colors, such that two arcs of the same color are "well-separated".
  \item Build a matrix deduced from the slack matrix by rescaling rows and adding approximately $n^{1/2}$ vectors, s.t. the "lampshade" argument is applicable (i.e. the entries for vertices and factes of the same arc are zero).
  \item Apply the "lampshade" argument and obtain the desired bound.
\end{enumerate}

Proof outline
\begin{itemize}
  \item Look at $M[V,F_c]$ separately for each color $c$ (try to bound its nonnegative rank by the number of arcs of that color)
  \item Build the matrix $K$ (by rescaling rows, s.t. for the same index separated vertices have larger scaled slacks than local vertices, and subtracting vectors, s.t. "diagonal" = 0)
  \item Bound the nonnegative rank of $K$ by "lampshade" argument (because diagonal = 0) 
  \item Lampshade argument: Facets and vertices are seperated -> surround vertices with polygon of constantly many vertices, which lie on the "positive slack" of all facets -> Convex combine all seperated vertices + Build slack vector for lampshade vertices -> Matrix rows are convex combination of constantly many vectors -> nonnegative rank constant
\end{itemize}