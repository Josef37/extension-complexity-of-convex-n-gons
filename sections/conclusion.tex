\section{Concluding Remarks}
% todo: formulate

In this paper we looked at the best bounds for the extension complexity of polygons currently known. 

For cyclic polygons $P$ we know $\xc(P) \in \Theta(n^{1/2})$.

For arbitrary polygons $P$, we only know $\xc(P) \in \Omega(n^{1/2}) \cap O(n^{2/3})$. So, there is still a gap to close.

Now the obvious question is: Which of these bounds is actually tight?

The authors of \cite{kwan2020extension} conjectured that cyclic polygons, as they "seem to represent quite a diverse cross-section of the space of all polygons" (p.\,3), have worst-case extension complexity. So $\xc(P) \in \Theta(n^{1/2})$.

On the other hand, the author of \cite{shitov2020sublinear} expected in Conjecture 61, that $\pc(n) = n^{1/2} \cdot \alpha(n)$ with unbounded $\alpha(n)$.

He reasons that the method developed in his paper doesn't seem to allow an $O(n^{1/2})$ upper bound for the worst-case $n$-gon complexity. As we saw, this means there is no subsequence of $\Omega(n)$ vertices, for which we could apply Theorem~\ref{theorem:glueing}.

And he also goes on explaining that Padrol proved in \cite{padrol2016extension} 
\begin{equation}\label{eq:wcc}
  \wcc(d,n) \geq 2\sqrt{dn-d}-d+1 ,
\end{equation}
where $\wcc(d,n)$ is the largest possible extension complexity of a polytope with $n$ vertices in a $d$-dimensional space. 
If the conjecture were false, the bound in \eqref{eq:wcc} would become asymptotically optimal for $d=2$, which is not expected, since for the \emph{correlation polytope} \cite{kaibel2015short} proved that $\wcc(m^2,2^m) \geq 1.5^m$. This is asymptotically much higher than $(\sqrt{2})^m$, which is the result of equation \eqref{eq:wcc} in that case.\\
That's why he does not expected \eqref{eq:wcc} to become asymptotically tight.

So the question of worst-case extension complexity of polygons remains exciting.
